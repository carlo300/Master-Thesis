\chapter{Stacks}
    \label{chapter:stacks}
    \section{Recalls of algebraic geometry}
        In this work we will use, when needed, classical algebraic geometry, where classical means \cite{Hart}. Let's just recall some definitions to warm up.
        
        \begin{defn}
            \label{defn:smooth_etale_classic}
            Let $f\colon \Spec A \to \Spec B$ a morphism in $\Aff$, the category of affine schemes, corresponding to a map of rings $B \to A$.
            \begin{itemize}
                \item The map $f$ is \emph{smooth} if it is flat, of finite presentation and $B$ is a $B \otimes_A B$-module (by the multiplication map) of finite Tor dimension. This last condition is equivalent as asking $\Omega^1_{B/A}$ to be a projective (= locally free) $B$-module.
                \item The map $f$ is \emph{étale} if it is smooth and, in particular, $B$ is a flat $B \otimes_A B$-module. This last condition is equivalent as asking $\Omega^1_{B/A} = 0$, the $B$-module of K\"{a}hler differentials.
            \end{itemize}
        \end{defn}
        Clearly étale implies smooth (precisely étale means smooth with relative dimension $0$). One intuitive way to think is to see smooth maps as the analogue of submersions in differential geometry, and étale maps as the analogue of local diffeomorphisms (inducing isomorphisms on tangent spaces and hence local diffeomorphisms).

        Of course these definitions are stable by composition and base change.
        For a more detailed exposition see \cite[Chapther~III, 10]{Hart}.

        % Grothendieck topologies
        We will assume to be known the notion of Grothendieck topology and the general notion of sheaf on a site, for which our main reference will be \cite{Vist:desc}. We will sometimes use pretopologies (i.e.\ set of arrows $\{U_i \to U\}$ covering $U$) and sometimes sieves (i.e.\ subfunctors of $h_U$), recalling that there is a canonical way to pass from one to the other (we have a one-to-one correspondence considering only saturated pretopologies).
        Let's now recall the definitions of fppf, fpqc and étale topology.
        \begin{defn}
            \label{defn:fpqc_morphism}
            Let $f\colon X \to Y$ a faithfully flat morphism of schemes. If $Y$ can be covered by open affine subschemes $\{V_i\}$ such that each $V_i$ is the image of a quasi-compact open subset of $X$, then we say that the map $f$ is \emph{fpqc}.
        \end{defn}
        See \cite[Proposition~2.33]{Vist:desc} for some equivalent characterizations.

        \begin{defn}
            \label{defn:grothendieck_topologies}
            Let's work in the category of affine schemes $\Aff$ (or also in the whole $\Sch$). 
            \begin{itemize}
                \item An \emph{étale} covering $\{U_i \to U\}_i$ is a jointly surjective collection of étale maps locally of finite presentation.
                \item An \emph{fppf} covering $\{U_i \to U\}_i$ is a jointly surjective collection of flat maps locally of finite presentation.
                \item An \emph{fpqc} covering $\{U_i \to U\}_i$ is a collection of morphisms such that $\coprod_i U_i \to U$ is fpqc (see \cref{defn:fpqc_morphism}).
            \end{itemize}
        \end{defn}

        Observe that the fpqc topology is finer than the fppf topology, which is finer than the étale topology, which is in turn finer than the Zariski topology.
        A lot of properties of morphisms are local on the codomain in the fpqc topology, see \cite[Proposition~2.36]{Vist:desc}.

    \section{Higher stacks}
        Let's consider the category of affine schemes $\Aff = \Comm^{\op}$ (in the following we will always use this equivalence implicitely) endowed with the étale Grothendieck topology, generated by coverings of $\Spec A$ of the type $\{\Spec A_i \to \Spec A\}_i$ such that any $\Spec A_i \to \Spec A$ is an étale map and the family of functors $\{- \otimes A_i\colon A-\Mod \to A_i-\Mod\}$ is conservative (this is an equivalent version of \cref{defn:grothendieck_topologies}). In an analogue way we consider the site $\Aff/X$ for a given $X$ affine scheme. We will assume to be based on $k \in \Comm$ (i.e.\ we implicitely consider the comma site $\Aff/\Spec k$) but, for readability, we will drop the comma notation.\newline
        We can then talk about sheaves on $\Aff$ and we'll focus on simplicial presheaves $SPr(\Aff) \simeq \sSet^{\Aff^{\op}}$, which we endow with the projective model structure (called \emph{global model structure}). This means that weak equivalences and fibrations are defined pointwise. One can prove this defines a cofibrantly generated model structure, proper and cellular. \newline
        We will now use the étale topology on $\Aff \simeq \Ho(\Aff)$ (using trivial model structure) to define a local model structure, obtained as a left Bousfield localization of the global structure. 
        Given a presheaf $F\colon \Aff^{\op} \to \sSet$ we can consider a presheaf of sets $\Aff \ni X \mapsto \pi_0(F(X))$, which we sheafify to get $\pi_0(F)$. Similarly, for $X \in \Aff$ and $s \in F(X)_0$, we define $\pi_j(F, s)$ to be the sheafification of the presheaf of groups \[(\Aff/X)^{\op} \ni (f\colon Y \to X) \mapsto \pi_j(F(Y), f^*(s)).\] 
        \begin{defn}
            \label{defn:classical_homotopy_sheaves}
            Using the same notation as above, the sheaves $\pi_0(F)$ and $\pi_i(F, s)$ are called the \emph{homotopy sheaves of $F$}.
        \end{defn}
        Observe that they are clearly functorial in $F$. To compute homotopy groups we can either choose a fibrant replacement in $\sSet$ and apply the classical combinatorial definition, or just consider the topological homotopy group of the geometric realization.
        \begin{defn}[Local model structure]
            \label{defn:local_structure_classic}
            Let $f\colon F \to F'$ be a map in $SPr(\Aff)$.
            \begin{enumerate}
                \item The map $f$ is a \emph{local equivalence} if $\pi_0(f)$ is an isomorphism of sheaves, as well as any $\pi_j(F, s) \to \pi_j(F', f(s))$, for all $X \in \Aff$ and $s \in F(X)_0$.
                \item The map $f$ is a \emph{local cofibration} if it is a global cofibration.
            \end{enumerate}
            Local fibrations are defined by lifting properties. This structure is called the \emph{local model structure} on $SPr(\Aff)$ (it can be proved it actually defines a model structure, see \cite{Blander:model}), and we will use it from now on.
        \end{defn}
        We can give a characterization of fibrant objects in $SPr(\Aff)$, thanks to a general theorem of \cite{DHI:hypercover}. We will need a general definition.%\todo{look at general stuff with sk and cosk}
        \begin{defn}[Hypercovering]
            \label{defn:hypercovering_classic}
            Given $X \in \Aff$, the data of a morphism $H \to X$ in $SPr(\Aff)$ (implicitely using Yoneda embedding on $X$) is called an (étale) \emph{hypercovering} if it satisfies:
            \begin{enumerate}
                \item for any integer $n$, the presheaf of sets $H_n$ is a disjoint union of representable presheaves \[H_n \simeq \coprod_i H_{n,i} \] for $H_{n,i} \in \Aff$ (Yoneda);
                \item for any $n \geq 0$ the morphism of presheaves of sets \[H_n \simeq H^{\Delta^n} \simeq \calHom(\Delta^n, H) \to \calHom(\partial\Delta^n, H) \times_{\calHom(\partial\Delta^n, X)} \calHom(\Delta^n, X) \] induces an epimorphism on associated sheaves.
            \end{enumerate}
            Here $\Delta^n$ and $\partial\Delta^n$ are considered as constant simplicial presheaves, while $\calHom$ is the presheaf of morphisms, valued in sets, between simplicial presheaves.
        \end{defn}

        \begin{example}
            \label{example:cech_nerve}
            Let $\{U_i \to X\}_{i \in I}$ be an étale covering in $\Aff$. Then the \v{C}ech nerve $H \in SPr(\Aff)$ of this covering (more classicaly denoted by the \v{C}ech nerve of $U = \coprod_i U_i$), given by \[H_n = \coprod_{(i_0, i_1, \dots,i_n) \in I^{n+1}} U_{i_0} \times_X U_{i_1} \times_X \dots \times_X U_{i_n} \] with face maps given by projections and degeneracies given by diagonal (implicitely using Yoneda as always), is a basic example of hypercovering.
        \end{example}
        We can restate the second condition of hypercoverings just by saying that given $Y \in \Aff$ and a commutative square
        \begin{diag}
            \partial\Delta^n \arrow[d] \arrow[r] & H(Y) \arrow[d] \\
            \Delta^n \arrow[r] & X(Y) = \textrm{const}\Hom(X, Y)
        \end{diag}
        in $\sSet$, then there exists a covering sieve $u \subset h_Y$ such that for any $f\colon U \to Y$ in $u(U)$, there exists a dashed lift 
        \begin{diag}
            \partial\Delta^n \arrow[d] \arrow[r] & H(U) \arrow[d] \\
            \Delta^n \arrow[r] \arrow[ur, dashed] & X(U) = \textrm{const}\Hom(X, U)
        \end{diag}
        making the diagram commutative. This is indeed a local lifting property, analogue to the lifting property characterizing acyclic fibrations in simplicial sets. Let's finally observe that, since $X$ is a $0$-truncated simplicial presheaf (i.e.\ all homotopy sheaves $\pi_i$ are zero for $i \geq 1$), the restriction $X^{\Delta^n} \to X^{\partial\Delta^n}$ is actually an isomorphism for $n > 1$, so that the second condition becomes only dependent on $H$ for $n > 1$ (requiring $H_n \to H^{\partial\Delta^n}$ to be an iso). 

        Given an hypercovering we define an augmented cosimplicial diagram in $\sSet$ \[F(X) \to \left([n] \mapsto F(H_n)\right) \] where $F(H_n) = \prod_i F(H_{n,i})$. 
        \begin{thm}
            \label{thm:classical_fibrant_obj}
            A simplicial presheaf $F$ over $\Aff$ is fibrant if and only if 
            \begin{enumerate}
                \item for any $X \in \Aff$, $F(X)$ is fibrant;
                \item for any $H \to X$ hypercovering, the natural map \[F(X) \to \holim_{[n] \in \Delta} F(H_n) \] is an equivalence in $\sSet$.
            \end{enumerate}
        \end{thm}
        \begin{proof}
            See \cite{DHI:hypercover}.
        \end{proof}

        If $F$ is a discrete presheaf, i.e.\ a presheaf of sets, then the second condition boils down to the usual descent condition.
        \begin{defn}
            \label{defn:stack_classic}
            An $F \in SPr(\Aff)$ is called a \emph{stack} if it satisfies the descent condition of \cref{thm:classical_fibrant_obj}. The homotopy category $\Ho(SPr(\Aff))$ is called the \emph{category of stacks} and its morphisms are denoted by $[F, F']$.
        \end{defn}
        \begin{remark}
            \label{remark:stack_sheaf_classic}
            Although for $F$ a presheaf of sets, the stack condition is equivalent to the set-theoretic sheaf condition, for a general simplicial presheaf $F$ being a sheaf of simplicial sets and being a stacks are two different concepts: indeed the homotopy limit can be different from the standard limit, and we have a more ``relaxed'' condition.
            Moreover, one can prove that if $F$ is groupoid-valued (as in the classic theory of $1$-stacks), then the homotopy limit boils down to the classic $2$-Ker condition, see \cite{Hennion:memoire}. More precisely, to pass from groupoid-valued stacks to simplicial stacks one just composes with the nerve functor (and obtains a $1$-truncated simplicial presheaf).
        \end{remark}
        Let's observe that stacks are really a generalization of sheaves, by the full embedding \[\mathrm{Sh}(\Aff) \to \Ho(SPr(\Aff)) \] considering any sheaf as a constant simplicial presheaf. This functor has a left adjoint $F \mapsto \pi_0(F)$.
        %\todo{Add examples and maybe stacks of stacks}
        \begin{remark}
            \label{remark:why_work_on_aff}
            To define stacks we work on the category of affine schemes $\Aff$, and not on the category $\Sch$ of general schemes, which is more usual in the classic context of stacks in groupoids. This doesn't affect at all the degree of generality of the theory: covering any scheme by affine opens we can reduce ourselves to only look at affine schemes, and we can obtain a ``bigger'' stack on $\Sch$ using the descent condition. This will remain valid also in the next chapters, where we will generalize this construction to the simplicial world.
        \end{remark}

        \subsection{\texorpdfstring{Structure of $SPr(\Aff)$}{Structure of the category of simplicial presheaves}}
            It will be useful to know a little better the category $SPr(\Aff)$, in particular its canonical $\sSet$-enrichment. The following reasoning will hold in great generality for the category of simplicial objects of any cocomplete category $\Cat$.

            \begin{defn}
                \label{defn:simplicial_enrichment}
                Let $K \in \sSet$ and $F \in s\Cat$, and define $F \otimes K$ by \[(F \otimes K)_n \coloneqq \coprod_{k \in K_n} F_n \in \Cat. \] Given $\phi\colon [n] \to [m] \in \Delta$ we define a map $\phi^*\colon (F \otimes K)_m \to (F \otimes K)_n$ by 
                \begin{diag}
                    \coprod_{k \in K_m} F_m \ar[r,"\coprod \phi^*"] & \coprod_{k \in K_m} F_n \ar[r] & \coprod_{k \in K_n} F_n 
                \end{diag}
                where the first map is induced by $\phi^*\colon F_m \to F_n$ and the second by $\phi^*\colon K_m \to K_n$.
            \end{defn}
            
            \begin{thm}
                \label{thm:simplicial_enrichment}
                Suppose that $\Cat$ is bicomplete (for example any category of presheaves of sets). Then with this bifunctor $- \otimes -\colon s\Cat \times \sSet \to s\Cat$, the category $s\Cat$ becomes a simplicial category with \[\IntHom_{s\Cat}(A, B)_n = \Hom_{s\Cat}(A \otimes \Delta^n, B). \]
            \end{thm}
            \begin{proof}
                See \cite[Chapter~II, Theorem~2.5]{GoeJar:simpl_hom}.
            \end{proof}
            Let's now state a version of Yoneda lemma valid for general simplicial presheaves $SPr(\Cat)$. Let's observe beforehand that the action of $\sSet$ on the simplicial category $SPr(\Cat)$ (using \cref{thm:simplicial_enrichment}) is given by \[F \otimes K\colon [n] \mapsto \coprod_{K_n} F_n \leadsto F \otimes K \simeq F \times K \] where $F \in SPr(\Cat)$, $K \in \sSet$ and in the last passage $K$ is considered as a constant presheaf. 
            \begin{lemma}
                \label{lemma:simplicial_yoneda_lemma}
                Let $\Cat$ be a category and $F \in SPr(\Cat)$ be a simplicial presheaf. For any $X \in \ob(\Cat)$ there is a natural isomorphism \[\IntHom_{SPr(\Cat)}(h_X, F) \simeq F(X) \] of simplicial sets, where $h_X = \mathrm{const} \Hom_{\Cat}(-, X)$ is the Yoneda discrete simplicial presheaf. 
            \end{lemma}
            \begin{proof}
                See \cite[Theorem~5.3.6]{Vez:seminar}.
            \end{proof}
            Let's now deduce a useful corollary for the case of $SPr(\Aff)$, endowed with the local model structure. It obviously can be generalized, being careful with fibrant and cofibrant objects.
            \begin{corollary}
                \label{corollary:simplicial_yoneda_lemma_pi0}
                Let $F \in SPr(\Aff)$ be a fibrant simplicial presheaf (hence a stack, according to our previous definition). For any $X \in \ob(\Aff)$ there is a natural isomorphism 
                \[\pi_0(\IntHom_{SPr(\Aff)}(h_X, F) ) \simeq \Hom_{\Ho(SPr(\Aff))}(h_X, F) \simeq \pi_0(F(X))  \] of sets.
            \end{corollary}
            \begin{defn}
                \label{defn:homotopy_pullback_classic}
                Let $F \longrightarrow H \longleftarrow G$ be a diagram of stacks in $\Ho(SPr(\Aff))$. We will denote by $F \times^h_H G$ the homotopy fiber product of some lift of such diagram in $SPr(\Aff)$ (this construction is thus not functorial in $\Ho(SPr(\Aff))$).
            \end{defn}
    \section{Geometric stacks}
        \label{section:geometric_stacks}
        Here we will give a new definition of schemes, and then of geometric $n$-stacks, that need to be thought as quotients of schemes. More precisely the basic idea is the following: we consider a representable stack $X$ and a groupoid object $X_1$ (which is itself a representable stack) acting smoothly on $X$, and then we consider the quotient. Sometimes $X_1$ needs not to be representable, but it can be itself a quotient of a representable stack, and this is the main motivation behind the recursive definition of geometric $n$-stacks (where we assume $X_1$ to be $(n-1)$-geometric). A more extensive explanation, although in a more general context, can be found at \cite[1.3.3]{ToVe:hag2}.

        Recall that any affine scheme $\Spec A$ can be identified, through the Yoneda map, to the presheaf \[\Spec B \mapsto \Hom(\Spec B, \Spec A) = \Hom(A, B) \] which is actually a sheaf for the étale topology (faithfully flat descent, see \cite[Theorem~2.55]{Vist:desc}), and hence it can be considered as a constant simplicial stack in $\Ho(SPr(\Aff))$. The Yoneda embedding \[ h\colon \Aff \to \Ho(SPr(\Aff))\] is fully faithful.
        \begin{defn}
            \label{defn:affine_scheme}
            Any stack isomorphic, in the homotopy category $\Ho(SPr(\Aff))$, to one $\Spec A$ (using Yoneda embedding) is called an \emph{affine scheme}, or a \emph{representable stack}.
        \end{defn}
        We can now define schemes.
        \begin{defn}\hfill
            \label{defn:stack_map_properties_classic}
            \begin{enumerate}
                \item A morphism $F \to \Spec A$ is a \emph{Zariski open immersion} if $F$ is a sheaf (i.e.\ $0$-truncated), $i$ is a monomorphism of sheaves and there exists a family of classical Zariski open immersions $\{\Spec A_i \to \Spec A\}_i$ such that the map \[\coprod_i \Spec A_i \to \Spec A \] factors through an epimorphism of sheaves to $F$.
                \item A morphism $F \to F'$ is a Zariski open immersion if it is locally so, i.e.\ for any affine scheme $\Spec A$ and any map $\Spec A \to F'$, the induced map \[F \times^h_{F'} \Spec A \to \Spec A \] is a Zariski open immersion as in the previous point.
                \item A stack $F$ is a \emph{scheme} if there exists a family of affine schemes $\{\Spec A_i\}_i$ with open immersions $\Spec A_i \to F$ such that the induced morphism of sheaves \[\coprod_i \Spec A_i \to F \] is an epimorphism. Such a family is called a \emph{Zariski atlas} for $F$.
                \item A morphism of schemes $F \to F'$ is called \emph{smooth} if it is ``locally smooth'', i.e.\ if there exist Zariski atlases $\{\Spec A_i \to F\}$ and $\{\Spec B_j \to F'\}$ such that we have commutative squares 
                \begin{diag}
                    F \ar[r] & F' \\
                    \Spec A_i \ar[u] \ar[r] & \Spec B_j \ar[u]
                \end{diag}
                with the downward morphism being (classically) smooth (here for any $i$ we find $j = j(i)$).
            \end{enumerate}
        \end{defn}
        Finally we are ready to define geometric stacks.
        \begin{defn}\hfill
            \label{defn:algebraic_stack_classic}
            \begin{enumerate}
                \item A stack $F$ is \emph{$(-1)$-geometric} if it is representable (i.e.\ an affine scheme).
                \item A morphism of stacks $F \to F'$ is \emph{$(-1)$-representable} if for any representable stack $X$ and any map $X \to F'$, the homotopy pullback $F \times^h_{F'} X$ is $(-1)$-geometric.
                \item A $(-1)$-geometric morphism $F \to F'$ is \emph{$(-1)$-smooth} if for any representable stack $X$ and any map $X \to F'$, the induced morphism $F \times^h_{F'} X \to X$ is a smooth morphism between representable stacks.
            \end{enumerate}
            Let $n > 0$ and assume the notions of $(n-1)$-geometric stack, morphism and smooth morphism to be defined. Then, by recursion on $n$, we can define the following.
            \begin{enumerate}
                \item An stack $F$ is $n$-geometric if there exists a family of maps $\{U_i \to F\}_{i \in I}$ such that 
                \begin{enumerate}[label=(\alph*)]
                    \item each $U_i$ is representable,
                    \item each map $U_i \to F$ is $(n-1)$-smooth,
                    \item the total morphism $\coprod_{i \in I} U_i \to F$ is an epimorphism.
                \end{enumerate}
                Such family is a \emph{smooth $n$-atlas}.
                %\item A stack $F$ is $n$-geometric if there exists a scheme $X \to F$, being a $(n-1)$-algebraic smooth epimorphism. Such a morphism is called a \emph{smooth $n$-atlas} for $F$.
                \item A morphism $F \to F'$ is $n$-representable if for any representable stack $X$ and any map $X \to F'$, the stack $F \times^h_{F'} X$ is $n$-geometric.
                \item An $n$-geometric morphism $F \to F'$ is $n$-smooth if for any representable stack $X$ and any map $X \to F'$, there exists a smooth $n$-atlas $\{U_i\}$ of $F \times^h_{F'} X$ such that each composite map $U_i \to X$ is smooth.
                %\item An algebraic stack/morphism/smooth morphism is a stack $F$/morphism $F \to F'$ which is $n$-algebraic/smooth for some $n$.
            \end{enumerate}
        \end{defn}
        Observe that, since Zariski open immersions are smooth, schemes are $0$-geometric stacks.
        One can prove, although non-trivially, that if $F$ is an $n$-geometric stack, then its diagonal is $(n-1)$-representable.
        Our definition makes sense, as justified by the following statement.
        \begin{prop}\hfill
            \label{prop:n_rep_bigger_n}
            \begin{enumerate}
                \item Any $(n-1)$-representable (resp.\ $(n-1)$-smooth) morphism is $n$-representable (resp.\ $n$-smooth).
                \item All $n$-representable (resp.\ $n$-smooth) morphisms are stable by isomorphisms, homotopy pullbacks and compositions.
            \end{enumerate}
        \end{prop}
        \begin{proof}
            See \cite[Proposition~1.3.3.3]{ToVe:hag2}.
        \end{proof}
        To conclude, let's state some other important properties of $n$-geometric stacks and $n$-smooth maps. 
        \begin{prop}
            \label{prop:n_geometric_local}
            Let $f\colon F \to G$ be a morphism of stacks, where $G$ is $n$-geometric. Suppose there exists a smooth $n$-atlas $\{U_i\}$ of $G$ such that each stack $F \times^h_G U_i$ is $n$-geometric. Then $F$ is also $n$-geometric.
            Furthermore, if each projection $F \times^h_G U_i \to U_i$ is $n$-smooth, then $f$ is also $n$-smooth.
        \end{prop}
        \begin{proof}
            The slogan of this statement could be that $n$-geometricity and $n$-smoothness are local on $n$-geometric targets. See \cite[Proposition~1.3.3.4]{ToVe:hag2} for the proof.
        \end{proof}
        \begin{prop}
            \label{prop:m_smooth_n_smooth}
            Let $f$ be an $n$-representable morphism. If $f$ is $m$-smooth, for $m \geq n$, then it is $n$-smooth.
        \end{prop}
        \begin{proof}
            See \cite[Proposition~1.3.3.6]{ToVe:hag2}.
        \end{proof}
        A vey important corollary is the following.
        \begin{corollary}
            \label{corollary:stability_n_geom_stacks}
            Let $n \geq 0$; then the full subcategory of $n$-geometric stacks in $\St(k) = \Ho(SPr(\Aff_{/k}))$ is stable by homotopy pullbacks and by disjoint unions.
        \end{corollary}
        \begin{proof}
            See \cite[Corollary~1.3.3.5]{ToVe:hag2}.
        \end{proof}

        We conclude with one last definition.
        \begin{defn}
            \label{defn:smooth_maps}
            A stack is a \emph{geometric stack} if it is $n$-geometric for some $n$.
            A morphism of stacks is \emph{smooth} (resp.\ \emph{representable}) if it is $n$-smooth (resp.\ $n$-representable) for some $n$.
        \end{defn}

        Geometric stacks can be, maybe more intuitively, described, as announced in the beginning, in terms of quotients by groupoid actions. For time issues we won't report here this point of view, which can be found at \cite[1.3.4, 1.3.5]{ToVe:hag2}.