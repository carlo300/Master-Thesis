\chapter{Introduction}
    In this work I studied the basic definitions of derived algebraic geometry, starting from the generalization of classical stacks in groupoids (which will be called, in a more modern way, $1$-stacks) to simplicial stacks to then introduce derived stacks and then study some concrete examples. The main reference has been undoubtedly \cite{ToVe:hag2}, which is where the approach to derived algebraic geometry with model categories comes from (studied in a much more general way).
    Not all the necessary background material has been covered here, for time (and space) issues: first of all the basic theory of model categories is assumed to be known to the reader, and only some more advanced definitions are recalled. The main references used for model categories are \cite{Hov:model} and \cite{Hirs:loc}. Also, the basic theory of simplicial sets is constantly used: a good reference is \cite{GoeJar:simpl_hom}. Basic facts about stacks in groupoids, using pseudofunctors or fibered categories, are also assumed to be known: the reference here is \cite{Vist:desc}. Finally, scheme theory contained in \cite{Hart} will be used, if needed.

    Here follows a more detailled abstract of this work. In \cref{chapter:model_categories} we recall some definitions coming from the theory of model categories, like what are simplicial and monoidal model categories, the main properties of the left Bousfield localization and of homotopy function complexes. These results are just stated: although the theory of model categories is interesting, we just need to know how to manipulate such objects (for example mapping sets, and their properties, will be heavily used in the rest of the work). 

    In \cref{chapter:stacks}, after a brief recall of some algebraic geometry definitions, we introduce simplicial presheaves and stacks (which will be the fibrant objects of the local model structure on $SPr(\Aff)$, for the étale topology, or, more simply, simplicial presheaves satisfying an homotopical descent condition). We also mention the link between classical stacks in groupoids and simplicial stacks: composing the former with the nerve functor we obtain a particular class of simplicial stacks (we just call them stacks from now on), whose homotopy groups $\pi_n$ are all trivial for $n \geq 2$. The descent condition with the homotopy limit, in this case, just gives back the more well-known $2$-descent condition: a more extensive explanation can be found in \cite{Hennion:memoire}. Finally we define, recursively, $n$-geometric stacks: the most intuitive way to think about them is using quotients; for example a $1$-geometric stack is obtained as a quotient of a scheme by a smooth action of a group scheme. The whole theory of schemes is, of course, embedded into the theory of stacks using the Yoneda lemma (indeed representable stacks are just isomorphic to affine schemes).

    In \cref{chapter:moduli_stack_bung} we take a little break to carefully study a classical example: the moduli stack of $G$-bundles on a scheme $X$ over $S$. The goal is to prove that, under suitable condition on $X$, it is an algebraic stack with a schematic diagonal map (using classical wording), covered by open substacks of finite presentation over $S$. This part is mainly a rewriting of \cite{wang:moduli}. 
    We use the classical language of pseudofunctors and $2$-categories: some definitions, like the $2$-fibered product, are recalled in the beginning of the chapter. The interest of this chapter, which can seem a bit unrelated from the rest of the work, is its final theorem, which will be used later to apply Lurie's representability criterion, \cref{thm:lurie_rep_criterion}. And also, it is a good example of how complex the problem of representability can be, and how non-trivial (and long and tedious) can be the process of finding the right open covering of a given stack.

    We finally begin to dive into the \emph{derived} world in \cref{chapter:derived_stacks}, where we start recalling some results of simplicial algebra, like the definition of simplicial modules and of stable simplicial modules (needed for the cotangent complex). We generalize then some classical definitions related to modules and maps (like flat, étale or projective) to our simplicial (homotopical) case. We finally define derived stacks: the idea is quite similar to the definition of stacks in \cref{chapter:stacks}, what changes here is that simplicial presheaves are taken on the category of $\dAff = \sComm^{\op}$, i.e.\ we don't use just rings but simplicial rings. We need to consider two successive left Bousfield localizations of $SPr(\dAff)$ (starting with the obvious projective model structure) to get to the analogue of the local model structure (w.r.t.\ the simplicial version of the étale topology), because we need to take into account the nontrivial model structure on $\sComm$. Finally, derived stacks are just fibrant objects in this model category, i.e.\ simplicial presheaves which preserve equivalences between simplicial rings and satisfy an homotopical descent condition. Again, we define $n$-geometric derived stacks in the same way as before. The basic objects here will be affine derived schemes, written as $\RSpec A$ using the model Yoneda lemma, \cref{prop:model_yoneda_lemma}. 
    Obviously stacks embed into derived stacks, since any ring is a discrete simplicial ring, and we can also restrict a derived stack only on classical rings: truncation and extension are the functorial way to pass from stacks to derived stacks. Finally we introduce the definition of cotangent complex (and of higher tangent spaces) of a derived stack, which can be thought as an homotopical generalization of the cotangent sheaf for smooth schemes, or equivalently as an homotopical corepresentative of derived derivations.

    We gathered enough theory and terminology to be able to see some concrete examples of derived stack: this is what is done in \cref{chapter:examples}. We mainly consider the derived stack of local systems on a simplicial set (or, equivalently, on a topological space) and the derived stack of vector bundles on a scheme, which is a particular case of a mapping derived stack. They are both generalizations of well known classical stacks: the classical part of the derived stack of vector bundles is $\Bun_{\GL_n}$, studied in detail in \cref{chapter:moduli_stack_bung}. We compute, in both cases, the cotangent complex at a point (and then globally) and finally, for the derived stack of vector bundles, we apply Lurie's representability criterion and prove it is $1$-geometric. 

    \section*{Notation and conventions}
        For us all rings are associative, unital and commutative. We will never mention universes, and totally ignore any related set-theoretic issue: the rigorous treating of the subject can be done, and can be found in \cite{ToVe:hag2}. 
        
        We will often abbreviate ``with respects to'' as w.r.t.\ as well as ``left lifting property'' as LLP.
        The category of simplicial sets $\sSet$ will be endowed with the Quillen model structure, and we will write $\Top$ to mean the category of compactly generated weak Hausdorff spaces with its Quillen model structure (so that $\sSet$ and $\Top$ are Quillen equivalent model categories). We will often write $X \in \Cat$ or $f\colon X \to Y \in \Cat$ to mean $X \in \ob(\Cat)$ or $f\colon X \to Y \in \mor(\Cat)$. Furthermore we will equally denote the hom sets in $\Cat$ either by $\Hom_{\Cat}(X, Y)$ or $\Cat(X, Y)$. Finally, we will both use the terms trivial and acyclic (which mean the same thing in the context of model category) as well as the words filtrant and directed (for colimits). 
